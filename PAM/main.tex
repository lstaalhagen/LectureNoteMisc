\documentclass[dvips,11pt,a4paper]{article}

\usepackage[danish]{babel}
\usepackage[utf8]{inputenc}
\usepackage{amsmath}
\usepackage{amssymb}

%%
%% Set non-standard dimensions for the page layout
%%
\setlength{\hoffset}{0pt}
\setlength{\voffset}{0pt}
\setlength{\oddsidemargin}{0pt}
\setlength{\evensidemargin}{0pt}
\setlength{\textwidth}{16cm}
\setlength{\topmargin}{-37pt}
\setlength{\textheight}{670pt}
\setlength{\footskip}{60pt}


\title{PAM-signaler i MATLAB}
\author{Lars Staalhagen\\{\small larst@fotonik.dtu.dk}}
\date{}

\begin{document}

\maketitle

\noindent{}Dette dokument beskriver brugen af foldningsfunktionen i MATLAB som en metode til at generere et PAM-signal, også kaldet et ''pulstog''.

Et PAM-signal i kontinuert tid kan helt generelt defineres via formel\footnote{Jf. også formel (5.1) i Knud J. Larsen, ''Introduktion til digital kommunikation'', 2017, hvor der dog er tale om, at informationssymbolerne, $a_k$ er defineret for alle $k$-værdier. I denne note vil $a_k$ dog kun være relevante for $k\ge{}0$.}~(\ref{eq:pamanalog})
\begin{equation}
v(t)=\sum_{k=0}^{\infty} a_{k} \cdot g(t-k \cdot T)
\label{eq:pamanalog}
\end{equation}
hvor
\begin{itemize}
\item $a_k$ er informationssymboler ($k=0,1,2,\ldots$).
\item $g(t)$ er pulsformen
\item $T$ er symbolperioden, dvs. tidsintervallet mellem to $a_k$-symboler skal sendes.
\end{itemize}

\noindent{}I digitale kommunikationssystemer er vi dog kun interesseret i signalerne til diskrete tidspunkter, dvs. de analoge signaler er blevet digitaliseret (dvs. samplet og kvantiseret) og vi er derfor kun interesseret i værdierne til tidspunkterne $t = n \cdot T_{s}$, hvor $T_s$ er samplingsintervallet og $n = 0,1,2,\ldots$. En tidsdiskret udgave af ligning~(\ref{eq:pamanalog}) fås derfor ved at erstatte $t$ med $n \cdot T_s$ i ligning~(\ref{eq:pamanalog}):
\begin{equation}
v(n \cdot T_{s})=\sum_{k=0}^{\infty} a_{k} \cdot g(n \cdot T_{s}-k \cdot T)
\label{eq:pamdiskret2}
\end{equation}

\noindent{}I praksis vil samplingsintervallet, $T_s$, være valgt så $T = m \cdot T_s$ ($m \in \mathbb{N}$) for en passende værdi af $m$. Herved kan ligning~(\ref{eq:pamdiskret2}) omskrives
\begin{align}
v(n \cdot T_{s}) &= \sum_{k=0}^{\infty} a_{k} \cdot g(n \cdot T_{s}-k \cdot m \cdot T_s) \\
 &= \sum_{k=0}^{\infty} a_{k} \cdot g((n - k \cdot m) \cdot T_{s}) 
\end{align}

\noindent{}Indføres derefter notationen $x_n$ som en kortere måde at skrive $x(n \cdot T_s)$ på fås:
\begin{equation}
v_n = \sum_{k=0}^{\infty} a_{k} \cdot g_{n - k \cdot m}
\label{eq:pamdiskret}
\end{equation}

\noindent{}Opskrives de første par værdier af $v_n$ til illustration fås:
\begin{align*}
v_0 &= a_{0}\cdot{}g_{0} + a_{1}\cdot{}g_{-m} + a_{2}\cdot{}g_{-2m} + \ldots \\
v_1 &= a_{0}\cdot{}g_{1} + a_{1}\cdot{}g_{1-m} + a_{2}\cdot{}g_{1-2m} + \ldots \\
v_2 &= a_{0}\cdot{}g_{2} + a_{1}\cdot{}g_{2-m} + a_{2}\cdot{}g_{2-2m} + \ldots \\ &\vdots
\end{align*}
dvs. helt generelt kan udtrykket for $v_n$ skrives som
\begin{equation}
v_{n} = a_{0}\cdot{}g_{n} + a_{1}\cdot{}g_{n-m} + a_{2}\cdot{}g_{n-2m} + \ldots{} 
\label{eq:v_n}
\end{equation}

\noindent{}Inden for matematikken defineres {\em foldningen} af to tidsdiskrete sekvenser som vist i ligning~(\ref{eq:mathconv}) (hvoraf den ene sekvens blot kan være den samme som $g_k$-sekvensen ovenfor)
\begin{equation}
y_n = \sum_{l=0}^{\infty} b_{l} \cdot g_{n-l}
\label{eq:mathconv}
\end{equation}

\noindent{}Det ses, at ligningerne~(\ref{eq:pamdiskret}) og~(\ref{eq:mathconv}) har næsten samme form, og da MATLAB har en indbygget funktion, kaldet \texttt{conv()}, der implementerer\footnote{I MATLAB vil der naturligvis være tale om, at både $b$- og $g$-sekvenserne har en endelig længde, men det er faktisk ikke nødvendigt i formel~(\ref{eq:mathconv}).}  ligning~(\ref{eq:mathconv}), er det nærliggende at forsøge at omskrive ligning~(\ref{eq:pamdiskret}), så den har samme form som ligning~(\ref{eq:mathconv}), så et PAM-signal i MATLAB kan genereres via MATLAB's \texttt{conv()}-funktion.

Opskrives på samme måde de første værdier af $y_n$ ud fra ligning~(\ref{eq:mathconv}) fås:
\begin{align*}
y_0 &= b_{0}\cdot{}g_{0} + b_{1}\cdot{}g_{-1} + b_{2}\cdot{}g_{-2} + \ldots + b_{m}\cdot{}g_{-m} + \ldots +  b_{2m}\cdot{}g_{-2m} + \ldots \\
y_1 &= b_{0}\cdot{}g_{1} + b_{1}\cdot{}g_{0} + b_{2}\cdot{}g_{-1} + \ldots + b_{m}\cdot{}g_{1-m} + \ldots +  b_{2m}\cdot{}g_{1-2m} + \ldots \\
y_2 &= b_{0}\cdot{}g_{2} + b_{1}\cdot{}g_{1} + b_{2}\cdot{}g_{0} + \ldots + b_{m}\cdot{}g_{2-m} + \ldots +  b_{2m}\cdot{}g_{2-2m} + \ldots \\
 &\vdots
\end{align*}
som generelt kan skrives som
\begin{equation}
y_n = b_{0}\cdot{}g_{n} + b_{1}\cdot{}g_{n-1} + b_{2}\cdot{}g_{n-2} + \ldots + b_{m}\cdot{}g_{n-m} + \ldots +  b_{2m}\cdot{}g_{n-2m} + \ldots
\label{eq:y_n}
\end{equation}

\noindent{}Ved at opskrive ligningerne~(\ref{eq:v_n}) og~(\ref{eq:y_n}) sammen (og tilføje nogle led med værdien 0 til udtrykket for $v_n$) fås:
\begin{align*}
v_{n} = a_{0}\cdot{}g_{n} &+ 0\cdot{}g_{n-1} + 0\cdot{}g_{n-2} + \ldots + 0\cdot{}g_{n-m+1}\\
&+ a_{1}\cdot{}g_{n-m} + 0\cdot{}g_{n-m-1} + 0\cdot{}g_{n-m-2} + \ldots + 0\cdot{}g_{n-2m+1}\\
&+ a_{2}\cdot{}g_{n-2m} + \ldots{} \\
y_{n} = b_{0}\cdot{}g_{n} &+ b_{1}\cdot{}g_{n-1} + b_{2}\cdot{}g_{n-2} + \ldots + b_{m-1}\cdot{}g_{n-m+1} \\
&+ b_{m}\cdot{}g_{n-m} + b_{m+1}\cdot{}g_{n-m-1} + b_{m+2}\cdot{}g_{n-m-2} + \ldots + b_{2m-1}\cdot{}g_{n-2m+1}\\
&+ b_{2m}\cdot{}g_{n-2m} + \ldots \\
\end{align*}

\noindent{}Det ses herved, at udtrykkene for hhv. $v_n$ og $y_n$ kan bringes til at stemme overens, dvs. at $v_n = y_n$ som ønsket, hvis $b_l$-sekvensen udregnes ud fra $a_k$ sekvensen på følgende måde:
\begin{align*}
b_{0}&=a_{0},\\
b_{1}&=b_{2}=\ldots{}=b_{m-1}=0,\\
b_{m}&=a_{1},\\
b_{m+1}&=b_{m+2}=\ldots{}=b_{2m-1}=0,\\
b_{2m}&=a_{2}\\
&\vdots
\end{align*}
hvilket også kan skrives som
%\[
%b_{l}=\begin{cases}
%a_{l/m} &\textrm{for }l=k\cdot{}m\textrm{ hvor }(k\in{}\mathbb{N}_{0}) \\
%0 &\textrm{ellers}
%\end{cases}
%\]
%eller
\[
b_{l}=\left\lbrace
\begin{array}{ll}
a_{l/m} & \textrm{for }l=k\cdot{}m\textrm{ hvor }(k\in{}\mathbb{N}_{0}) \\
 & \\
0 & \textrm{ellers}\\
\end{array}
\right.
\]

\noindent{}Dvs. $b_{l}$ sekvensen dannes ud fra $a_{k}$ sekvensen ved at der "indsættes" $m-1$ stk. 0'er mellem de oprindelige værdier, dvs. som
\[
a_0,0,0,\ldots,0,a_1,0,0,\ldots,0,a_2,0,0,\ldots
\]

\noindent{}Dvs. det er muligt at generere/udregne et PAM-signal i MATLAB på flg. måde:
\begin{enumerate}
\item Lav en \emph{opsamplet} version, kaldet $au_l$ (som svarer til $b_l$-sekvensen ovenfor), af $a_k$-sekvensen ved at indsætte $m-1$ 0-værdier efter hver værdi fra $a_k$-sekvensen.
\item Udregn foldningen af $au_l$ og $g_l$ via MATLAB's \texttt{conv()}-funktion
\end{enumerate}

\noindent{}Dvs. flg. MATLAB kode kan benyttes til at generere et pulstog:
\begin{verbatim}
a = ...;  % Informationssekvens
m = ...;  % Antallet af samples pr. symbolperiode
g = ...;  % Udtryk for pulsformen
au = [a; zeros(m-1,length(a))];
au = reshape(au,1,[]);
v = conv(au,g);
\end{verbatim}

\noindent{}

\end{document}
