\section{Routing in general}

\subsection{Definitions}
\begin{itemize}
%
\item\emph{Routing} -- Determining a path from a source to a destination.
%
\item\emph{Switching} -- The process in a network node of receiving a piece of information on an input interface and sending the information out on an output interface
%
\item\emph{Router} -- A node/device in a network that performs routing and switching. Usually associated with layer 3 (the network layer) in the OSI reference model.
%
\item\emph{Routing protocol} -- A protocol for exchanging information between routers, to permit these to perform routing
%
\item\emph{Routing table} -- A table maintained in a router that can lists the next--hop as a function of the destination identity.
%
\item\emph{Source routing} -- Routing where packets contain the full path from the source to the destination in their header.
%
\item\emph{Hop--by--hop routing} -- Routing, where the packets only contain the identity of the destination node, and where the routers then determine the next--hop from the routing table.
%
\item\emph{Static and dynamic routing} -- In static routing, the paths through the network are determined at network initialization. These paths are not changed unless required, e.g. by a change of the network topology. In dynamic routing, the paths are continually updated.
%
\item\emph{Distributed vs. Centralized routing} -- In centralized routing, there is one node that is responsible for all path calculations. In distributed routing, every router makes a decision.
%
\item\emph{Flat vs. Hierarchical routing} -- In flat routing, the path calculations take every part of the network into account, while in hierarchical routing, the network is divided into different domains. In the latter case, this means that two kinds of routing must be handled: intra--domain routing and inter--domain routing.
%
\end{itemize}

\subsection{Some special routing principles}

\subsubsection{Flooding}

\subsubsection{Deflection routing}

