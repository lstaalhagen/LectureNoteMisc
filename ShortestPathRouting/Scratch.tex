In the
first case, the edges include an arrowhead to indicate the
direction of the communication; this edge is then said to be a
\emph{directed} edge. A graph with directed edges are referred to, as a directed graph.

A graph is said to be \emph{directed} if it includes one or more directed edges; otherwise the graph is \emph{undirected}.


Every edge in the graph is associated with a metric (a number)
that represents the "length" of that edge, so that the path from a
source node to a destination node has a total length of the sum of
the length of the edges that are part of the path. In this sense,
the link--metrics are said to be \emph{additive}.

The function that is used to generate the metric is chosen by the
network operator to reflect his preferences, for instance that the
paths determined tend to favor low--delay links and tend to avoid
high--cost links.

Note
that the results of these algorithms are not just the shortest path
from a given source node to a given destination node, but instead a
set of shortets--paths, e.g. the shortest--paths from a given
source--node to \emph{all} other destination nodes.

